\documentclass[10pt,a4paper]{article}
\usepackage[utf8]{inputenc}
\usepackage[scale=0.7,vmarginratio={1:2},heightrounded]{geometry}

\usepackage[numbers]{natbib}

% link support in pdf
\usepackage[colorlinks,allcolors=blue,breaklinks = true]{hyperref}

% images in pdf
\usepackage{graphicx}
\graphicspath{{Bilder/}}
% multiple images
\usepackage{subfigure}
% images float in text
\usepackage{float}

% url support
\usepackage{url}
% glossarie support
\usepackage[acronym, automake]{glossaries}
\makeglossaries
\loadglsentries{glossary}

% math libs
\usepackage{amsmath}
\usepackage{amssymb}
\usepackage{amstext}
\usepackage{amsfonts}
\usepackage{mathrsfs}

% Code formating
\usepackage{listings}
\usepackage{color}

\definecolor{dkgreen}{rgb}{0,0.6,0}
\definecolor{gray}{rgb}{0.5,0.5,0.5}
\definecolor{mauve}{rgb}{0.58,0,0.82}

\lstset{frame=tb,
  language=Java,
  aboveskip=3mm,
  belowskip=3mm,
  showstringspaces=false,
  columns=flexible,
  basicstyle={\small\ttfamily},
  numbers=none,
  numberstyle=\tiny\color{gray},
  keywordstyle=\color{blue},
  commentstyle=\color{dkgreen},
  stringstyle=\color{mauve},
  breaklines=true,
  breakatwhitespace=true,
  tabsize=3
}

%Metadata
\title{TBD}
\author{Simon Hischier}
\date{April 2018}

\begin{document}

%Titlepage
\begin{titlepage}
%\maketitle
\centering
\vspace{1cm}
	{\scshape\LARGE Fachhochschule Luzern HSLU \par}
	\vspace{1cm}
	{\scshape\Large Studiengang Digital Ideation, Bachelor \par}
	
	{\scshape\Large 4. Semester\par}
	\vspace{1.5cm}
	{\huge\bf Procedural Generation in the age of AI\par}
	
	\vspace{10cm}
	{\Large Simon Hischier\par}
	\vfill

% Bottom of the page
	{\large \today\par}
\end{titlepage}

%Table of Contents Page
\renewcommand{\contentsname}{Inhalt}
\tableofcontents
\newpage

%First real page
\section{Abstract}
bliblablup todo

\textbf{Fragestellung}: How does content creation in games differ between \gls{pcg} and \gls{ml} content creation?

\section{Definitions}
\subsection{Content}
Content in games can refer to various parts. Some known areas include level generation (dungeons, 

\subsection{Procedural Content Generation}
The term \gls{pcg} ..

\subsection{Artificial Intelligence}
In 1956 the Book Automata Studies\cite{McCarthy1956} layed the ground work for \gls{ai} and the Dartmouth summer research project on artificial intelligence marked the key event "to nail the flag to the mast." McCarthy is credited for coining the phrase “artificial intelligence” and solidifying the orientation of the field\cite{moor2006dartmouth}. The name \gls{ai} was used ever since for various applications. Ever since this key event \gls{ai} is defined based on the goal that is tried to beeing achieved. Bernard Marr lists them as following:\cite{Marr2018}
\begin{enumerate}
\item Build systems that think exactly like humans do ("strong AI") 
\item Just get systems to work without figuring out how human reasoning works ("weak AI")
\item Use human reasoning as a model but not necessarily the end goal
\end{enumerate}
Marr referes with "strong AI" and "weak AI" to the paper written by John Searle where he defines a strong \gls{ai} of beeing able to think and have a mind and a weak \gls{ai} that can only act like it thinks and has a mind. The paper is also known for the "Chinese Room" argument\cite{Searle1980}. More terms for a \gls{ai} classification exist like Artificial General Intelligence (AGI) and the Artificial Superintelligence (ASI). These classifications are not further defined in this work.

Bernard Marr continues to list the various definitions for \gls{ai}. While the dictionaries list \gls{ai} as a definition, companies lack a clear definition and Marr extrapolates a definition from the companies research field. The dictionary definitions and his extrapolated definitions are listed here:
\begin{enumerate}
\item \textbf{The English Oxford Living Dictionary} "The theory and development of computer systems able to perform tasks normally requiring human intelligence, such as visual perception, speech recognition, decision making, and translation between languages."
\item \textbf{Merriam-Webster}
\begin{enumerate}
\item A branch of computer science dealing with the simulation of intelligent behavior in computers.
\item The capability of a machine to imitate intelligent human behavior.
\end{enumerate}
\item \textbf{The Encyclopedia Britannica} "artificial intelligence (AI), the ability of a digital computer or computer-controlled robot to perform tasks commonly associated with intelligent beings."
\item \textbf{Amazon} defines it as "the field of computer science dedicated to solving cognitive problems commonly associated with human intelligence, such as learning, problem solving, and pattern recognition."
\item \textbf{Google AI} "create smarter, more useful technology and help as many people as possible"
\item \textbf{Facebook AI Research} "advancing the file of machine intelligence and are creating new technologies to give people better ways to communicate."
\item \textbf{IBM}'s three areas of focus are "AI Engineering, building scalable AI models and tools; AI Tech where the core capabilities of AI such as natural language processing, speech and image recognition and reasoning are explored and AI Science, where expanding the frontiers of AI is the focus."\cite{Marr2018}
\end{enumerate}
This work is using the definition of the term \gls{ai} as listed in \textit{The Encyclopedia Britannica}. The fields described in this work can be categorized in the 3. objective listed by Marr "Use human reasoning as a model but not necessarily the end goal"\cite{Marr2018}

\subsection{The Difference between AI, Machinelearning, Deeplearning}
\label{sec:AIMachinelearningDeeplearning}
blibla \cite{MichaelCopeland2016} TODO: research definitions
\begin{figure}[H]§
	\includegraphics[width=\textwidth, height=\textheight, keepaspectratio]{Deep_Learning_Icons_R5.png}
	\caption{Levels of AI as Image\cite{MichaelCopeland2016}.}
\end{figure}

\section{Motivation}
%The early days of PC games were all about \gls{pcg} TODO: https://ieeexplore.ieee.org/abstract/document/5593333/
Games that are meant to be replayed a lot benefit from Procedural Generation.

\section{Convolution Neural Network}
TODO \gls{cnn}

\section{Recurrent Neural Networks}
As explained in \autoref{sec:AIMachinelearningDeeplearning} \gls{ml} is a broad term and includes a variety of models. The \gls{rnn} are networks for tasks where we need some kind of persistance. If we want to classify videoframes the network should have some kind of consistency.\cite{Olah2015} A network should persist the last seen data and not reclassify items every frame. Reclassifying without previous context could lead to different recognitions in every frame for the same object.

\section{Long short-term memory}
\gls{rnn} are good for persisting very recent information. Sentences are a great example: "Ships are built to float on \textit{water}". The \gls{rnn} is great in filling the end of this sentence. Problems arise when the information is needed a lot later. The more information is inbetween the contextual references the more unreiable a basic \gls{rnn} gets. Books for example can have references on the last page to the very beginning. For such tasks a \gls{lstm} model is the perfect fit. The \gls{lstm} network was introduced by Hochreiter Sepp and Urgen Schmidhuber\citep{Hochreiter1997}. A \gls{lstm} is a specialized version of a \gls{rnn} which is designed for these kind of tasks. Almost all \gls{rnn} tasks can be achieved with a \gls{lstm} \gls{rnn}\cite{Olah2015}. 
to read: $https://en.wikipedia.org/wiki/Long_short-term_memory$
$https://colah.github.io/posts/2015-08-Understanding-LSTMs/$

\section{The artistic vision and the generation}\label{sec:visionVSgeneration}
Generating content for games is a fundamental artistic choice for gamedevelopers. The generation in various forms is linked to a decrease in the artistic vision. Designers have to step away from micro-controlling gameparts like environment, shapes, colors, enemy behaviours etc. Therefore games do include \gls{pcg} in different ways and in various depths. Big studios tend to stick to more controlled experiences and have more (human-)resources to ensure this vision. We define a list of various depths of \gls{pcg}:
\begin{enumerate}
\item \textbf{No generation} (ex. \textit{Super Mario Bros} (Nintendo, 1985) where everything was handplaced, drawn and animated as explained in \textit{Super Mario Bros}. Level 1-1\cite{EurogamerMiyamotoInterview})
\item \textbf{Content generation in the game making phase.} (ex. \textit{The Elder Scrolls Oblivion} (Bethesda Softworks, 2006) used \gls{pcg} to generated most of the world before the artists curated it.\cite{PCGamerCarterInterview} An example of a widely used \gls{pcg} algorithm middleware for game studios is SpeedTree\cite{SpeedTree}
\item \textbf{Gameplay (partially) definded or influenced} by \gls{pcg} such as the sidequests for \textit{The Elder Scrolls Skyrim} (Bethesda Softworks, 2011) which were endlessly generated\cite{Bertz2011} or Castles in \textit{Rogue Legacy} (Cellar Door Games, 2013) which are generated procedurally but the game has some kind of continuosity and progress on top of the castle runs.\cite{Stanton2013}
\item \textbf{Games almost completely generated} ex. \textit{Dwarf Fortress} (Dwarf Fortress, 2006) doesn't stop at the map generation. It starts out generating the history of this world and everything that happened before.\cite{Champandard2012}
\end{enumerate}
Games and even game genres do fall into these different levels of \gls{pcg}. A major role for this classification of games and game genres is the depth of artistic controll or lack therefore. Games that do rely more on \gls{pcg} tend to focus more on the fun gameplay rather then an intriguing story and complex characters.

\section{Procedural Content Generation and AI}
As explained in \hyperref[sec:visionVSgeneration]{The artistic vision and the generation} there are various genres and games falling into different levels of automation and generation. Due to the type of gameplay and game mechanics linked to \gls{pcg}, games with high levels of generation can easily be identified. We hope to blur the lines more and create \gls{pcg} games that are less distinguishable from handcrafted counterparts. By extending the \gls{pcg} with \gls{ai} we hope to have a more natural and handcrafted feel to \gls{pcg} type games. As \gls{ml} gets used more in games, the applications starts to vary further.

\section{Applications of Procedural Approaches}
\textbf{TO DO}

\section{Applications of AI systems}
\textbf{TO DO}

\subsection{Face representation in games}
An area where \gls{ai} is much more effective then handwritten procedrual approaches is face reconstruction and face mapping. Webcam feeds have no depth information and creating 3D faces from photos is a laborintensive work. Actual research shows \gls{ml} is capable of reconstruction and position maping 3D avatars from single images. It's robust, fast and stable\cite{Feng2018}. A large area in the games industry is avatar faces. Massive Multiplayer Games tend to have elaborate avatar creaton tools to customize the game character. Games like \textit{Arma} (Bohemia Interactive, 2013) are using the users voice in multiplayer sessions and try to synchronize the avatars lips with the players spoken words. Online chats with avatar representations such as \textit{VRChat} (VRChat Inc., 2017) have immersive VR worlds where players can walk and talk in a virtual environment. For streaming platforms like \textit{Twitch} (Amazon, 2011) a webcam feed is part of the majority of streams. All these different games, genres and platforms would benefit from some kind of reconstruction of the players face. \textit(Arma) could have the real users faces represented in real-time in the game, non human avatars could use \gls{ai} to map the users lips and expressions on to the virtual avatars. Streamer could have their faces as 3D models or have an avatar instead of a live camera feed.

\subsection{evolution}
asdf

\subsection{game mechanics}
ai game mechanics?

\subsection{art etc.}
ex. color palettes
style transfer

\subsection{assistance}
Dota 2 plus assistance

\subsection{text to speech}
\cite{Shen2017} text to speech and "A Neural Parametric Singing Synthesizer" for singing
\cite{Luan2018} painting with extra detail

\subsection{Game AI}
Something almost all games have is computer players controlled by \gls{ai}. There are a variety of different solutions to mimic human behaviour. The more complex a game is the harder it is to write comprehensible game \gls{ai}. Games like Go remain too hard to calculate a winning strategy. There are simply too many complex situations and too many possible paths to explore. In this field \gls{ai} really showed its power and Alpha Go defeating the world champion \textbf{TODO Paper ref} was a major step in \gls{ai}. While Alpha Go did need more hardwarepower then the usual gaming computer provides, OpenAI demonstrated a less powerhungry \gls{ai} for \textit{Dota 2} (Valve Corporation, 2013). At the International Tournament 2017 OpenAI demonstrated an \gls{ai} that beat Pro Players at the game reliably.\cite{Openai2017} The rules were more strict then a normal game of Dota 2 but given enough time and resources, an \gls{ai} that can beat Professional teams is likely. These types of games are a big challenge to program computer players for because they allow for wacky situations, situations never seen before and asymmetric or incomplete information about the other players. A strong field for \gls{ml} is creating \gls{ai} that always remains and adapts to the level of the human player. This is almost impossible to program by hand or with other approaches.

\subsection{Anti-Cheating}
ref valve anti cheat

\section{shortcomings of AI}
\textbf{TO DO}
, generation \gls{rpg}s started to rely more on generation and artists curation of content to fill the initial world.
-even more lack of control, very big input array
-same problems as with \gls{pcg} because a more complex understanding is needed for an artistic vision across levels

\section{Procedural Content Generation and Machine Learning}
\textbf{TO DO}
The different strengths of \gls{pcg} and \gls{ml} suggest, that \gls{ml} is an extension in the content generation toolbox rather then a replacement.

super mario level 1-1 teaching ...

\gls{ai} is a new type of content generation with overlapping areas to \gls{pcg} and classic game \gls{ai}s. It can go from a \gls{pcg} replacement to games revolving around an \gls{ai} (Alien game colonial marines?)

\section{Conclusion}

?
ai is harder to controll then \gls{pcg} and depending on the area is a lot more labor intensive. But has fields where \gls{pcg} is inappropriate.
?

\section{OLD:}
\section{TensorFlow.js}
As of April 2018 Google released a new JavaScript \gls{ml} library called \gls{tensorflowjs}. \gls{tensorflowjs} is build on deeplearn.js and can be used for a broad variety of \gls{ml} tasks. The library allows to train and run models in the webbrowser. Models can be pre-trained on a server or offline computer and then be used on the website.

\section{Challenges}
various challenges:
\begin{itemize}
\item level is playable
\item Levels getting harder
\item build up, learn new thing and then master it
\end{itemize}


\section{Goal}
The goal of this work is to generate maps for games. Maps from "Super Mario" or "Super Meat Boy" contain mostly solid or empty blocks. With that knowledge, the levels can probably be represented in an easy to read text format instead of an image format. The text can probably be feed into an \gls{ml} system and new maps can be generated with the assistance of a \gls{ml} model.

\section{Referenzen und Akronyme}

\printglossaries

%\renewcommand{\refname}{myBibliography}
\bibliography{myBibliography}
%\bibliographystyle{unsrtnat}
%\bibliographystyle{plainnat}
\bibliographystyle{unsrt}
%\bibliography{myBibliography}

%list the figures and tables in contents
%\addcontentsline{toc}{section}{\listfigurename}
%\addcontentsline{toc}{section}{\listtablename}

%print list
\listoffigures
\listoftables

%\nocite{*}

\end{document}