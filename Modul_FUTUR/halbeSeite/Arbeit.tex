\documentclass[10pt]{article}
\usepackage[utf8]{inputenc}


\usepackage[numbers]{natbib}

\bibliographystyle{plainnat}
\usepackage[colorlinks,allcolors=blue,breaklinks = true]{hyperref}

\usepackage{graphicx}
\usepackage{url}
\usepackage{float}
\usepackage[acronym, automake]{glossaries}
\usepackage{subfigure}

% math libs
\usepackage{amsmath}
\usepackage{amssymb}
\usepackage{amstext}
\usepackage{amsfonts}
\usepackage{mathrsfs}

% Code formating
\usepackage{listings}
\usepackage{color}

\definecolor{dkgreen}{rgb}{0,0.6,0}
\definecolor{gray}{rgb}{0.5,0.5,0.5}
\definecolor{mauve}{rgb}{0.58,0,0.82}

\lstset{frame=tb,
  language=Java,
  aboveskip=3mm,
  belowskip=3mm,
  showstringspaces=false,
  columns=flexible,
  basicstyle={\small\ttfamily},
  numbers=none,
  numberstyle=\tiny\color{gray},
  keywordstyle=\color{blue},
  commentstyle=\color{dkgreen},
  stringstyle=\color{mauve},
  breaklines=true,
  breakatwhitespace=true,
  tabsize=3
}

%configs
\graphicspath{ {Bilder/} }
\makeglossaries
\loadglsentries{glossary}

%Metadata
\title{2D levelcreation with Tensorflow}
\author{Simon Hischier}
\date{Juni 2017}

\begin{document}

%Titlepage
\begin{titlepage}
%\maketitle
\centering
\vspace{1cm}
	{\scshape\LARGE Fachhochschule Luzern HSLU \par}
	\vspace{1cm}
	{\scshape\Large Studiengang Digital Ideation, Bachelor \par}
	
	{\scshape\Large 4. Semester\par}
	\vspace{1.5cm}
	{\huge\bf TBD\par}
	
	
	\vspace{10cm}
	{\Large Simon Hischier\par}
	\vfill

% Bottom of the page
	{\large \today\par}

\end{titlepage}

%Table of Contents Page
\renewcommand{\contentsname}{Inhalt}
\tableofcontents
\newpage

%First real page
\section{Abstract}
TBD

\section{TensorFlow.js}
As of April 2018 Google released a new JavaScript \gls{ml} library called \gls{tensorflowjs}. \gls{tensorflowjs} is build on deeplearn.js and can be used for a broad variety of \gls{ml} tasks. The library allows to train and run models in the webbrowser. Models can be pre-trained on a server or offline computer and then be used on the website.

\section{Goal}
The goal of this work is to generate maps for games. Maps from "Super Mario" or "Super Meat Boy" contain mostly solid or empty blocks. With that knowledge, the levels can probably be represented in an easy to read text format instead of an image format. The text can probably be feed into an \gls{ml} system and new maps can be generated with the assistance of a \gls{ml} model.

\section{Referenzen und Akronyme}

\printglossaries

\bibliographystyle{unsrt}

\bibliography{myBibliography}

\end{document}